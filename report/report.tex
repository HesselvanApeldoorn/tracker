\documentclass[a4paper,10pt]{article}
\usepackage{graphicx}
\usepackage{color}
\usepackage{float}
\usepackage{amsmath}
\usepackage{amsfonts}
\usepackage{amssymb}
\usepackage{graphics}
\usepackage{listings}
\usepackage{graphicx}
\usepackage{verbatim}
\usepackage{url}
\usepackage{hyperref}
\usepackage[square, numbers]{natbib} % Bibliography citation framework (in the style "[9]")
\usepackage[intoc]{nomencl} % nomenclature
\usepackage[toc,page]{appendix} % Easier creation of appendices
\usepackage{soul}
\usepackage{cleveref}
\graphicspath{{img/}} % Specifies the directory where pictures are stored

\topmargin -1.5cm
\oddsidemargin -0.04cm
\evensidemargin -0.04cm
\textwidth 16.59cm
\textheight 21.94cm 
\parskip 7.2pt
\parindent 0pt

\definecolor{grijs}{rgb}{.92,.92,.92}

\lstset{
	language=c,
	basicstyle=\footnotesize,
	showstringspaces=false,
	numbers=left,
	numberstyle=\footnotesize,
	stepnumber=1,
	numbersep=5pt,
	backgroundcolor=\color{grijs},
	showspaces=false,
	showtabs=false
	commentstyle=\itshape,
	tabsize=3,
	postbreak=,
	breaklines=true
}

\makeatletter
\def\@xobeysp{ }
\makeatother


\makenomenclature
\renewcommand{\nomname}{Glossary}


\crefformat{footnote}{#2\footnotemark[#1]#3}

\begin{document}
\title{Ubiquitous Computing Project}
\author{Antonis Gkortzis (s2583070) \& Mark Hoekstra(s2015366) \& Hessel van Apeldoorn(s1881140)}
\date{\today}
\maketitle
\section{Project Description}
For the Ubiquitous Computing course we develop a tracking app for Android. This app tracks the location of the user continuously. It has the following features:
\begin{itemize}
\item Show locations you've been to on Google Maps (for the last hour/day/week)
\item Allow the user to mark locations. E.g. the Bernoulliborg being school and the ACLO being sports. The user can then see how much time he/she has spent doing something in a certain place.
\item Show statistics. Show how many kilometres you've traveled in a certain time, show which places you visited frequently, show the time you've spent there.
\item Show weather using an API
\end{itemize}

The tracking data will be used to make an estimation of the kind of person that is using the tracking app.
\begin{itemize}
 \item As a classification of the person using our app we will use the following: study (E.g. going to the BernoulliBorg a lot implies CS/AI), sleep pattern (E.g. being active/moving at night means you like to party), important locations (based on the time you are at a location, guess what this location is used for)
\item The state of world (both time and place) will aid us in finding the previously named classifications
\end{itemize}


\end{document}
